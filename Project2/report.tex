\documentclass[11pt,a4paper]{article}

\usepackage{anysize}
\usepackage{amsmath} % Required for some math elements 
\setlength\parindent{0pt} % Removes all indentation from paragraphs

\title{Project 2: PKI and TLS} % Title
\author{William Phan, Simon Wessel}
\date{\today} % Date for the report

\begin{document}
\maketitle % Insert the title, author and date

\setcounter{secnumdepth}{0}

\subsection{Assignment 2}
\textbf{2.1.} Serial number: 15772073244864896046 (0xdae1a90d5b9d5c2e)\\
\textbf{2.2.} Subject and issuer are the same.\\
\textbf{2.3.} Signing algorithm: RSA.\\
\textbf{2.4.} Hashing algorithm: sha256.\\
\textbf{2.5.} 65537\\
\textbf{2.6.} Subject key identifier:
     57:25:86:90:20:D1:D7:7E:28:91:67:CB:2E:2C:28:77:ED:0F:55:CD\newline
     \indent Authority key identifier:
     keyid:57:25:86:90:20:D1:D7:7E:28:91:67:CB:2E:2C:28:77:ED:0F:55:CD\\
     Basic constraints:
     CA:TRUE
\\
\subsection{Assignment 3}
\textbf{3.2.} It is the subject key identifier (and authority as it's self-signed) of the root certificate.\\
\textbf{3.3.} No need for a new private key. The certificates are different.\\
\textbf{3.6.} No, however it's important that the client has the server's root CA marked as trusted, and vice versa.\\
\\\\
\subsection{Assignment 5}
The distributor of the Java implementation decides what goes into the default TrustStore. Oracle points out that one has \\to be on control over the location where cacerts and jssecacerts are stored. In adition, one should always set the javax.net.ssl.trustStore property of secure clients and servers to maintain control over trusted certificates.
$https://access.redhat.com/documentation/en-US/Fuse_MQ_Enterprise/7.1/html/Security_Guide/files/SSL-SysProps.html$
\subsection{Assignment 6}
\subsection{Assignment 7}
\subsection{Assignment 8} %Initial connect: Syn packages.
\subsection{Assignment 9}
\end{document}