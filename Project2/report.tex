\documentclass[11pt,a4paper]{article}

\usepackage{anysize}
\usepackage{amsmath} % Required for some math elements 
\setlength\parindent{0pt} % Removes all indentation from paragraphs

\title{Project 2: PKI and TLS} % Title
\author{William Phan, Simon Wessel}
\date{\today} % Date for the report

\begin{document}
\maketitle % Insert the title, author and date

\setcounter{secnumdepth}{0}

\subsection{Assignment 2}
\textbf{2.1.} Serial number: 15772073244864896046 (0xdae1a90d5b9d5c2e)\\
\textbf{2.2.} Subject and issuer are the same.\\
\textbf{2.3.} Signing algorithm: RSA.\\
\textbf{2.4.} Hashing algorithm: sha256.\\
\textbf{2.5.} 65537\\
\textbf{2.6.} Subject key identifier:
     57:25:86:90:20:D1:D7:7E:28:91:67:CB:2E:2C:28:77:ED:0F:55:CD\newline
     \indent Authority key identifier:
     keyid:57:25:86:90:20:D1:D7:7E:28:91:67:CB:2E:2C:28:77:ED:0F:55:CD\\
     Basic constraints:
     CA:TRUE
     
\subsection{Assignment 3}
\textbf{3.2.} It is the subject key identifier (and authority as it's self-signed) of the root certificate.\\
\textbf{3.3.} No need for a new private key. The certificates are different.\\
\textbf{3.6.} No, however it's important that the client has the server's root CA marked as trusted, and vice versa.

\subsection{Assignment 5}
The distributor of the Java implementation decides what goes into the default TrustStore. Oracle points out that one has to be on control over the location where cacerts and jssecacerts are stored. In adition, one should always set the javax.net.ssl.trustStore property of secure clients and servers to maintain control over trusted certificates.
\\{\scriptsize$https://access.redhat.com/documentation/en-US/Fuse_MQ_Enterprise/7.1/html/Security_Guide/files/SSL-SysProps.html$}
\subsection{Assignment 6}
The purpose of the RootCA is to put restrictions on who can sign and what gets signed. This is important to maintain trust in the RootCA. In addition to the RootCA passphrase, the private key is also needed.\\
The purpose of the trust stores is to control whom to trust. The server and client use separate trust stores, and hence separate passphrases. The key tool forces the creator of the trust store to choose a passphrase.\\
The key stores store the certificates. When creating
For instance, if CA's key is not protected with a password, anything can be signed with\\
-------------------\\
The server and client each need to remember three passwords; TrustStore, KeyStore and key. It is possible to use the same passphrase for the key store as for the key, thus only needing two passwords. Using a KeyStore is considered a security enhancement, since keys and certificates cannot be acessed or manipulated without the KeyStore passphrase. This is important as the holder of a certificate, I want to be sure that the certificate is only used when I say so. Another benefit is that many files (certificate and key) are stored in one place, and are thus easier to manage.
\subsection{Assignment 7}
CertificateExpiredException (?)java.security.cert.CertificateExpiredException: NotAfter: Tue Mar 10 12:44:41 CET 2015
\subsection{Assignment 8} 
\begin{itemize}
\item \textbf{Key exchange method and packets}\\The first three packets are standard TCP connection establishment. The client sends \textit{ClientHello} to which the server responds with \textit{ServerHello, Certificate, ServerKeyEchange, HelloDone}. The client sends \textit{ClientKeyExchange}. \textit{EncryptedHandshakeMessage}s are exchanged in accordance with the Diffie-Hellman algorithm, which the automatically chosen cipher suite uses.
\item \textbf{Certificate information}\\
The server certificate can be found in the \textit{ServerHello} package. The server sends its own certificate before sending the CA certificate.
\item \textbf{Cipher suite}\\
When switching from automatically choosing \texttt{TLS\_ECDHE\_RSA\_WITH\_AES\_128\_CBC\_SHA256} to
\texttt{TLS\_RSA\_WITH\_AES\_128\_CBC\_SHA256}, the packets look somewhat different. The server doesn't send \textit{ServerKeyExchange}. Instead, the client sends \textit{ClientKeyExchange}. This difference is due to the handshake algorithms used. The latter example uses RSA, and thus only the client has to send an encrypted message to the server. The Diffie-Hellman algorithm used in the former example relies on both parties exchanging parts of what later becomes the key.
\end{itemize}

\subsection{Assignment 9}
\begin{itemize}
\item \textbf{Key exchange method and packets}\\ 
Same as above, but the server also sends a request of certificate in the hello package. Depending on the cipher suite, the client will then send its certificate to the server together with the key exchange.
\item \textbf{Certificate information}\\
The server sends its certificate to the client along with a request, to which the client will respond with a certificate of its own.
The order of the certificates from the client is the same as the server, the client's certificate appears before the CA.  
\end{itemize}
\end{document}